\externaldocument{../main.tex}

In this chapter I will present the countless opportunities that come from a merge between AI and the
upcoming 6G network. Clearly the opportunities depend on the architecture of the network and are
example specific but I am going to try and condense the structure of the various models proposed in the very recent literature into a series of shared features.

The idea of implementing AI as an agent in the field of 6G networking is rooted into the want for
the creation of the most versatile personal assistant in commerce. Attempts have been made in the
past on the consumer level with clearly inferior models (e.g. "Siri" or "Cortana") but now the
objective is to create a seamless and encompassing ecosystem that can adapt and provide the user
with information on his own self that goes beyond anything we can currently imagine. An example that
often recurs in the literature to explain what the potential of AI, is to have a set of non-invasive sensors that can provide with discretely precise measurements of health data (bpm,
glicemy, pressure, etc...) and having a model on a device, like a smartphone, that is able to understand
whether the values are in the norm and eventually provide pieces of advice to make them better. This would have
to be inserted in a framework that is also able to reliably understand the user's attitude and
context to avoid annoying push notifications.

The question that I will try to answer now is what is required to have a network that can handle
users, provide them with very custom experiences and react to situations and stimuli automatically
without the need of human intervention.

\bigskip
\noindent
END:
\phantomsection
\label{ssec:end}

Going from the ground-up we can build the network architecture starting with the sensors, the end
devices, basically a Fog of computational devices of all matters and sizes that are able to do
computation, generate data or process information. Less than the 1\% of these nodes is going to be
capable of running models and usually they will be extremely simple in nature, probably the only LLM
available would be the ones in the order of <1B.

The Fog is supposedly a very heterogeneous mass of computing devices. Probably the most important
characteristic of this element of the hierarchy is the variance of computational power of the
devices. Two different classes of devices came to mind when thinking about the Fog structure:

\underline{Sensors} - which are capable of handling only the measurement of certain parameters and some limited computations (usually they are battery-powered devices).

\underline{Non-sensing devices} - this category is quite broad but I would consider all of the personal
devices to be part of it, anything that the user can interact with and is connected to the 6G network should be considered a bigger node in the Fog and should be capable of running AI models.

The Fog is supposedly a very heterogeneous mass of computing devices, probably the most important
characteristic of this element of the hierarchy is the variance of computational power of the
devices. In the following I will list all of the devices that I think will be part of the network.

\bigskip
\noindent
CONTROLLER:
\phantomsection
\label{ssec:controller}

A Controller is also named Edge node in the recent literature and is basically a device or a set of
devices that have much more compute power than the end-nodes that they manage.

Its function is to be able to handle more complex inference and model training, such controllers
should have enough power to do inference on classic models like GPT-3 and, with some tricks, should
also be to undergo training and re-training of such models.

Considering the literature edge nodes or controllers are considered to be larger downstream nodes inside the
network capable of handling a complex inference using the results coming from smaller models at the
end of the network, the idea provided in \cite{pga}, was to be able to do an automatic crash report
for cars using LLMs running locally on the car or on the phone and aggregating the initial results
at edge level.

In other cases the LLM controller is used to understand the user's necessities and modifies
pre-existing models accordingly, like in \cite{ai4ci}. Another solution seems to be more suited for a RLM
approach, using a RLM trained to do efficient dispatching of resources \cite{llm6G}.

If we consider a more industry-driven example, even in the field of heavy machinery or energy
production, it's very useful to have a system able to take a decision extremely rapidly based on a
certain situation. Modern control systems are able to take action and make corrections rapidly and
precisely, providing excellent results and avoiding costly mistakes, having an LLM that is instead able
to make more "management-oriented" decisions in real time and based on data might provide useful to avoid
energy or time wastes.

\bigskip
\noindent
CLOUD:
\phantomsection
\label{ssec:cloud}

Last but not least it's important to take into consideration the fact that the bigger models that
are used as a foundation for all of the other models that can be deployed or cached in the
controller initially have to be trained somehow and somewhere.

If we consider the GPT-3 model, which is the one that is the name that appears the most in recent literature, it would be necessary to have more than 1000 GPUs running non-stop for more than
4 months to have the complete training of the 175B parameters that compose the model \cite{gaisnet}. This task would be impossible for the low power systems deployed at the end or even at the edge, therefore
it's necessary to leverage the power of big datacenters positioned at the center of the network
hierarchy.
Clearly, retraining GPT-3 is not an interesting task, but it's a way to show tha magnitude of the
task that has to be accomplished.

Essentially the datacenters would be used to train models (using some tricks that I will introduce
in the next section), store data and do important inference on extremely big datasets.

\bigskip
It was surprising, while writing this section, to notice how many similarities are there between the
architecture that I just drafted and the structure of 5G and B5G networks, if we forget about the
technical challenges of the upgrade again for a second we can see how the move from 5G is just going
to be a matter of reusing what was already implemented, upgrading the network structure and adding
AI.

The cloud section is already taken care of, since our current infrastructure basically runs on cloud
services and many and more have been created, thrived and ended (if we think about services like
Google Stadia, which offered a gaming platform running over the cloud or the ChatGPT service that
allows the interaction with the GPT cloud over the cloud). Simplicity is the key to success, the
architecture is nothing new or revolutionary, but the idea behind it is, and the solutions found to
solve some of the problems that I have swept under the rug to avoid unnecessary difficulties are
just as fascinating.

In the next session I will go through the various challenges that we will be facing when the moment
comes to start implementing the fusion of the service architecture of 5G with the AI agents and why
the upgrade is not as easy as it might look like in the above explanation.
