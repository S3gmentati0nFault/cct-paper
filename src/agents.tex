\externaldocument{../main.tex}

Mobile communication technologies have been a standard evolving since the very first
generation, released in 1981. Every ten years a new standard is formulated and
developed to propel the industry forward, updates have not been necessarily industry shifting, as an
example, the move from 3G to 4G was more a performance bump \cite{baeldung3g}, meanwhile the passage to 5G and 6G
represented and will represent a big leap forward in terms of network capabilities, the mobile
network and cloud infrastructure will be so tightly interconnected that will basically become one.

Currently, researchers from all fields and developers are studying and pushing the boundaries of
network technology to move from theoretical study to actual practice, the first real world tests of
the new network should be expected not earlier than 2030. According to Nokia\cite{nokiabell} the six
key aspects of the future generation of mobile communication technologies can be summarized in the
following subsections.

\bigskip
\noindent
NEW SPECTRUM TECHNOLOGIES:
\phantomsection
\label{ssec:spectrum-technologies}

5G, in essence, increased the datarates for all users in mobility both upstream and downstream,
alongside a complete re-organization of the network putting the final nail in the coffin of an
hardware network exclusively deicated to phone connectivity and calling, by using new spectral
wavelengths and a new type of antenna that took advantage of millimeter waves (wavelength in the
realm of the millimeters), the available spectrum bands were:
\begin{itemize}
	\item 24 - 71 GHz which is high band 5G.
	\item 2.5 - 4.9 GHz which is the mid band 5G.
	\item 600 - 2600 MHz which is the low band 5G.
\end{itemize}
6G is thought to be implemented to use more wisely the space of millimeter waves and go beyond 71
GHz allowing for a higher spectral efficiency, the new standard will incorporate the 5G spectrum
standard, widen it, and take better advantage of the pre-existing bands. The structure of the
spectrum is still being studied but the foundations seem to be the following:
\begin{itemize}
	\item 470 - 690 MHz extreme wide area coverage band.
	\item 600 - 2600 MHz wide area coverage band.
	\item 2.4 - 4.9 GHz urban band.
	\item 7 - 20 GHz urban subsection band.
	\item 24 - 71 GHz local hotspot band.
	\item > 92 GHz will be the extremely short range communication system and is currently being
	      called the sub-TeraHertz band.
\end{itemize}
It's easy to see how the concept of dividing frequences into sub-bands based on geographical area
coverage can be extended to the
architectural model in order to achieve specific results, I will be using this division to go
deeper into my architectural paradigm in \ref{sec:opportunities}.

\bigskip
\noindent
AI AND MACHINE LEARNING:
\phantomsection
\label{ssec:ai-ml}

On the particular aspects of AI and Machine Learning we have many diverse weapons that can be put to
the service of the network or to the service of the users depending on the end goal that needs to
be achieved.

AI for the end user comes especially in the form of the so called LLM or Large Language Model. This
machine learning models work on text and are trained in a way that they are able to predict, inside
a token sequence, the next most probable token, tokens are essentially sequences of words that can contain
one or more words \cite{wikiLLM}.

In the last couple of years models like BingAI, GPT-4, LLaMa, etc... Have seen a spike not only in
their usage but also in the capabilities of the deployed models. The multimodal
nature of some of the bigger models can be leveraged to allow a more reliable and detailed
resolution of the problems at hand and the use of techniques like fine-tuning, grounding or prompt
engineering approaches like "Chain of Thought" and "Retrieval Augmented Generation" can turn the devices in the hand of end users into
powerful assistants that can handle many tasks both on-demand and autonomously.

Private automation, robotics and healthcare are just a couple of the fields that will be most
impacted by the addition of LLMs very close to the edge of the network. In the following I will be
considering LLMs especially because of their importance in the current technological panorama.

\bigskip

LLM lifecycle can be summarized as follows:
\begin{enumerate}
	\item Pre-training: During the pre-training procedure the model undergoes a general training
	      procedure on very large unlabeled datasets, preparing it to handle user requests
	      leveraging a general (foundational) knowledge \cite{gaisnet}.
	\item Fine-tuning: Fine-tuning is an optional procedure that consists of aligning the
	      pre-trained network with the user's necessities throught a partial or total retrain.

	      Fine-tuning can be both complete, in which all parameters are updated,
	      and parameter-efficient, in which only a subset of the parameters is updated; in this case
	      the backbone is generally frozen and only a very small percentage of the total
	      parameters is re-trained, to achieve this LoRA and other compression techniques should be employed, I will go through an introduction to LoRA in section \ref{sec:technical-limitations}. Effort is being put into discovering new and better ways to do parameter efficient fine-tuning, full network fine-tuning is not as interesting of a research branch because of the security implications linked to the topic \cite{gaisnet}.

	\item Inference: Terminals input unlabeled data in the pre-trained and optionally fine-tuned model and
	      the network generates an output.
\end{enumerate}

The LLMs that we are in the habit of using at the moment actually do not undergo the fine-tuning
procedure, services like ChatGPT and Bing-AI use different tricks that allow the models underneath
to perform better in specific scenarios without the need of a true fine-tuning procedure.
The OpenAI service ChatGPT uses context and prompt (Prompt fine-tuning or prompt-engineering), to
enhance the model's capabilities \cite{openAI}; Microsoft's BingAI has a different approach to the
matter, undergoing a "grounding" process where, upon request, the model first fetches context
information from the internet from a reputable souce (like Wikipedia) regarding the key points of
the question and then uses the retrieved information alongside the user's prompt to generate a response\cite{zerotohero}.

AI for the network is going to be a collection of machine learning algorithms that handle all of the
infrastructural necessities of the network: secuirity, resource management, model splitting, model
migration to follow users' mobility.

\bigskip
\noindent
SECURITY AND TRUST:
\phantomsection
\label{ssec:security-trust}

The changes in the spectrum distribution of the network and the use of AI in the network forces
research to face a series of extremely important questions linked to private data security and
individual security.

Giving AI more power and control means that it's necessary to be able to
control its operation and make sure that is, not only, behaving according to specification, but also
not being fed polluted data during the training or fine-tuning process by malicious third parties. Having
extremely-high-frequency bands in the spectrum will expose communications to easy
eavesdropping, it's therefore necessary to create security procedures that allow communications to
remain private even when using these frequency bands.

With this particular aspect in mind, research
efforts are being made to see if AI models can be used to make internet connections more secure,
one of the propositions made was to allow AI models to change encryption schemes on the fly based on
real-time security analysis results \cite{6ainets}.

\bigskip
\noindent
SENSING NETWORK:
\phantomsection
\label{ssec:sensing-network}

The use of Machine Learning and AI techniques coupled with a very high number of sensors will allow
the nodes of the network to sense the environment around them, this is an extremely useful technique
to generate a Digital Twin of the real world, storing precious information about the surroundings
and generating information through LLM inference.

Regarding the above Minrui Xu et al. \cite{pga} wrote a paper discussing how the multimodality of
today’s LLMs alongside a wide array of sensors could be used to generate automatically car crash
reports, in the field of automated driving, that can be further refined by sending them to a more
powerful LLM deployed at the edge of the network.

\bigskip
\noindent
EXTREME CONNECTIVITY:
\phantomsection
\label{ssec:extreme-connectivity}

The level of the services that are probably going to be brought to 6G require extremely fast and
extremely low-latency communication, therefore the Ultra-Reliable and Low-Latency Communication
service will be refined and improved to serve the new standard and to bring the latency below one
millisecond.

\bigskip
\noindent
NEW ARCHITECTURAL MODELS:
\phantomsection
\label{ssec:architectural-models}

6G is thought to work alongside cloud and allow users to experience it differently, the paradigm of
having big datacenters at the top of the cloud that handle all of the computation will be shifted
towards an extremely distributed architecture centered around data management pipelines that involve
only the devices in the hierarchy that are of use to the computation itself.

This means that the 6G network will be deploying, alongside the big datacenters at the top of the
cloud, a series of intermediate stations that grow in computational power the more the data goes up
in the hierarchy. Such intermediate nodes will be the ones that handle all of the computation
associated to the infrastructure management, they will also manage all of the LLM-associated tasks
linked to users (inference, fine-tuning, training, etc...).

\medskip
In the next sections I’ll be diving deeper in the opportunities, challenges and solutions associated
with the deployment of LLMs at the edge of the network.
