\externaldocument{../main}

There is no denying that AI is the buzzword of the moment and will probably remain at the top of the
charts for many months and many years to follow, the process that lead us here was definitely
difficult and the interest in the technology has shifted over the decades between the utmost
confidence and the deepest discomfort. Thanks to how deeply the landscape has changed over the last
couple of years, the definition of what the next technical standard for wireless networks has begun
its formation.

As stated in \cite{6ainets} 6G "is poised to revolutionize the landscape by delivering
unprecedented speed, reliability and security". This is a big promise, users will not be facing the
next incremental improvement that they are so used in seeing with consumer electronics, they will be
encountering a true revolution in the way they interact with the network. Alongside big promises,
though, come big challenges: How can we compress an AI model inside the limited resources of the
nodes at the edge of the network? How can we make inference with it in a useful fashion? How can we
assure that the network is functioning correctly and the model is correctly trained?

These are just a couple of the questions that we could be trying to ask ourselves and answer about
the standard that is being written down for 6G. Due to the clear vastness of the topic, for this
paper I chose to delve into some paper reviewing alongside some considerations about the specific
topic of Artificial Intelligence integration with the network, in particular in the next sections I
will go through the following concepts: \ref{sec:ai-agents} in which I will be explaining what types
of AI agents are being used in the field of Wireless networks research, \ref{sec:opportunities} in
which I will be introducing the different oppportunities that AI brings to the table, especially in
the field of 6G and edge computing, \ref{sec:technical-limitations} in this section I will be
explaining which are the biggest technical limitations for the implementation of LLMs at the edge of
wireless networks, I'll show some different architectures that have been proposed in literature and
I'll try to synthetize my own version based on the various influences; last but not least I will
introduce some challenges in \ref{sec:future-challenges} that are particularly interesting in a
network tightly integrated with one or more AI agents.
