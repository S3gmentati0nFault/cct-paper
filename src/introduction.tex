\externaldocument{../main}

There is no denying that AI is the buzzword of the moment and will probably remain at the top of the
charts for months and years to follow, the process that made AI what it is now has definitely been difficult
and the interest in the technology has repeatedly shifted between the utmost confidence and the deepest discomfort.

Every decade a new Wireless Communication Technology is released, back in 2020 5G was being
released, granting with its new antennae and its efficient spectral use a higher velocity both downstream
and upstream. Recently the research and development process has begun for the next Wireless
Communication standard. 
As stated in \cite{6ainets}, 6G is "poised to revolutionize the landscape by delivering
unprecedented speed, reliability and security". This is a big promise, users will not be facing the
next incremental improvement that they are so used in seeing with consumer electronics, they will be
encountering a true revolution in the way they interact with the network.

6G is infact meant to marry Wireless Networks and AI creating a chimera capable of speed,
customization and integration. This new standard is meant to bring to life an infrastructure that
will be able to support the immense requirements of the upcoming field of IoT.

All of this sounds incredibly promising on paper but, as it is known, big projects need to overcome even
bigger challenges: How can one put AI in the hands of the users and make it work reliably? How can
one make sure that large models behave correctly and reply in a timely fashion? How can one
assure that the network is functioning correctly and the model is correctly trained? How can we
handle the ever pressing problem of data security?

These are just a couple of the questions that we could be trying to ask ourselves and answer about
the standard that is being written down for 6G. Due to the sheer size of the topic, for this
paper I chose to delve into some paper reviewing alongside some considerations about the specific
topic of Artificial Intelligence integration with the network, in particular in the next sections I
will go through the following concepts: \ref{sec:ai-agents} in which I will be explaining what types
of AI agents are being used in the field of Wireless networks research, \ref{sec:opportunities} in
which I will be introducing the different oppportunities that AI brings to the table, especially in
the field of 6G and edge computing, \ref{sec:technical-limitations} in this section I will be
explaining which are the biggest technical limitations for the implementation of LLMs at the edge of
wireless networks, I'll show some different architectures that have been proposed in literature and
I'll try to synthetize my own version based on the various influences; last but not least I will
introduce some challenges in \ref{sec:future-challenges} that are particularly interesting in a
network tightly integrated with one or more AI agents.
