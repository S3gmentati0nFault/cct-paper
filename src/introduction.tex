\externaldocument{../main}

The impressive development of AI technologies that followed the publication of the notorious article
"Attention is all you need"\cite{attention} has pushed research teams to try and incorporate LLM
agents and similar models into anything that they can get their hands on, from consumer electronics
to cars, aerospace and industrial applications AI is going to be a pervasive part of the human
lifestyle from now on.

Every ten years the mobile communication standard changes and a new iteration is released, in 2020
5G was released and now the research and development process has begun for the next Mobile communication standard.
As stated in \cite{6ainets}, 6G is "poised to revolutionize the landscape by delivering
unprecedented speed, reliability and security". This is a big promise, users will not be facing the
next incremental improvement that they are so used in seeing nowadays with consumer electronics,
a revolutionary way of interacting with the network is apparently waiting for us less than ten year
from now.

6G is in fact meant to marry Wireless Networks and AI creating a chimera capable of speed,
customization and integration. This new standard is meant to bring to life an infrastructure that
will be able to support the immense requirements of the upcoming field of IoT.

All of this sounds incredibly promising on paper but, as it is known, big changes need to overcome even
bigger challenges: How can one put AI in the hands of the users and make it work reliably? How can
one make sure that LLMs behave correctly and reply in a timely fashion? How can one
assure that the network is functioning correctly and the model is correctly trained? How can we
handle the ever pressing problem of data security?

These are just some of the questions that we could be asking ourselves when it comes to 6G. Due to the sheer size of the topic, for this
paper I chose to delve into some paper reviewing alongside some considerations about the specific
topic of Artificial Intelligence integration with the network, in particular in the next sections I
will go through the following concepts: \ref{sec:ai-agents} in which I will be giving a very brief
overview of the 6G network and then move on to explaining what types
of AI agents are being used in the field of Wireless networks research, \ref{sec:opportunities} in
which I will be introducing the different opportunities that AI brings to the table, and I will
give some pointers as to what components are essential in the 6G architecture, \ref{sec:technical-limitations} in this section I will be explaining which are the biggest technical limitations for the implementation of LLMs at the edge of wireless networks, furthermore I will go through some solutions that have been provided to the above
problems; last but not least I will
introduce some challenges in \ref{sec:future-challenges} that are particularly interesting in a
network tightly integrated with one or more AI agents.
