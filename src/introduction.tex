\externaldocument{../main}

The impressive development of AI technologies that followed the publication of the notorious article
"Attention is all you need"\cite{attention} has pushed research teams to try and incorporate LLMs
and other such models into every aspect of human life. From consumer electronics to cars, aerospace
and industrial applications AI is going to be a pervasive part of society from now on.

Every ten years the mobile communication standard changes and a new iteration is released, 2020
paved the way to 5G era, and now the research and development process has begun for the next mobile communication standard.
As stated in \cite{6ainets}, 6G is "poised to revolutionize the landscape by delivering
unprecedented speed, reliability and security". This is a big promise, users will not be facing the
next incremental improvement that they are so used to seeing nowadays with consumer electronics,
a revolutionary way of interacting with the network is apparently waiting for us less than ten years
from now.

6G is in fact meant to marry Wireless Networks and AI creating a chimera capable of speed,
customization and pervasiveness. This new standard is meant to bring to life an infrastructure that
will be able to support the immense requirements of the upcoming field of IoT.

\bigskip

All of this sounds incredibly promising on paper but, as it is known, big changes need to overcome even
bigger challenges: How can one put AI in the hands of the users and make it work reliably? How can
one make sure that LLMs behave correctly and reply in a timely fashion? How can one
assure that the network is functioning correctly and the model is correctly trained? How can we
handle the ever pressing problem of data security?

These are just some of the questions that one could be asking about 6G research. Due to the sheer size of the topic, for this
paper I chose to delve into some paper reviewing alongside some considerations about the specific
topic of Artificial Intelligence integration with the network, in particular in the next sections I
will go through: \ref{sec:ai-agents} in which I will be giving a very brief
overview of the 6G network and then move on to explaining what types
of AI agents are being used in the field of Wireless networks research, \ref{sec:opportunities} in
which I will be introducing the different opportunities that AI brings to the table, and I will
give some pointers as to what components are essential in the 6G architecture, \ref{sec:technical-limitations} in this section I will be explaining which are the biggest technical limitations for the implementation of LLMs at the edge of wireless networks, furthermore I will go through some solutions that have been provided to the above problems; in the last section \ref{sec:future-challenges} I will dedicate myself to drawing some
conclusions and pointing at some research topics that might be of active interest.
